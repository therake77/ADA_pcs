\documentclass[10pt]{beamer}
\usepackage[utf8]{inputenc}
\usepackage{amsmath}
\usepackage{graphics}
\usepackage{listings}
\usepackage{courier}

\usetheme{Copenhagen}

\title[PC1]{Programación dinámica}
\subtitle{Práctica calificada II}
\author{Jarem Villalobos\inst{1} \and Joaquín Aynaya\inst{2}}
\date[22/09/2025]{Lunes 22 de septiembre de 2025}
\logo{\includegraphics[height = 1.0cm]{logoUNI.png}}

\begin{document}
    \frame{\titlepage}
    \begin{frame}
        \frametitle{Table of contents}
        \tableofcontents
    \end{frame}
    \section{Longest palindrome Sequence}
    \begin{frame}
        \frametitle{Longest palindrome Sequence: Definición}
        El siguiente algoritmo encuentra la longitud de la subsecuencia máxima de palíndromos que se pueden encontrar 
        dentro de una secuencia de caracteres.
        Para ello usa un matriz n x n, siendo n la longitud de la secuencia dada
    \end{frame}
    \begin{frame}[fragile]
        \frametitle{Longest palindrome Sequence: Pseudocódigo}
        \begin{lstlisting}[basicstyle=\ttfamily\scriptsize,tabsize=4]
Procedimiento Subsecuencia(s)
	n <- longitud(s)
	crear matriz de longitud (n)x(n), inicializada en 0
	Para i desde 0 hasta n-1:
		dp[i][i] <- 1
	FinPara
	Para m desde 2 hasta n:
		Para i desde 0 hasta n-m:
			j <- i+m-1
			Si s[i] = s[j]:
				Si m = 2:
					dp[i][j] <- 2
				Sino:
					dp[i][j] <- 2+dp[i+1][j-1]
				FinSi
			Sino:
				dp[i][j] <- max(dp[i+1][j],dp[i][j-1])
			FinSi
		FinPara
	FinPara
	devolver dp[0][n-1]
FinProcedimiento
        \end{lstlisting}
    \end{frame}
    \begin{frame}
        \frametitle{Longest palindrome Sequence: Complejidad}
        \begin{itemize}
            \item Inicializar la matriz nxn tiene como complejidad temporal $\mathcal{O}(n^2)$
            \item Se ejecuta un bucle for n veces, teniendo una complejidad $\mathcal{O}(n)$ 
            \item El bucle anidado: \begin{itemize}
                \item El bucle externo se ejecuta desde 2 hasta n, dando n-1 iteraciones -> O(n).
	            \item El bucle interno se recorre desde 0 hasta n-tamaño -> O(n)
	            \item Cada operación dentro de los bucle tiene complejidad temporal O(1)
            \end{itemize}
            \item El bucle anidado tiene complejidad $\mathcal{O}(n^2)$
        \end{itemize}
        \begin{equation*}
            T(n) = \mathcal{O}(n^2) + \mathcal{O}(n) + \mathcal{O}(n^2) = \mathcal{O}(n^2)
        \end{equation*}
    \end{frame}
    \section{Ordenación de multiplicación de matrices}
    \begin{frame}
        \frametitle{Ordenación de multiplicación de matrices: Definición}
        El siguiente algoritmo encuentra el orden optimo de multiplicación de una 
        secuencia de multiplicaciones matriciales. Para ello recibe un arreglo de las 
        dimensiones de cada matriz, siendo los elementos i-1 e i el numero de filas y 
        columnas respectivamente de la matriz i.
    \end{frame}
    \begin{frame}[fragile]
        \frametitle{Ordenación de multiplicación de matrices: Pseudocódigo}
        \begin{lstlisting}[basicstyle=\ttfamily\scriptsize,tabsize=4]
Procedimiento MenorNumeroDeProductos(p):
    n <- longitud(p - 1)
    Crear matriz dp de longitud (n+1) x (n+1), inicializada en 0

    Para m desde 2 hasta n:
        Para i desde 1 hasta n - m + 1:
            j <- i + m - 1
            dp[i][j] <- infinito
            Para k desde i hasta j - 1:
                costo <- dp[i][k] + dp[k+1][j] + p[i-1] * p[k] * p[j]
                Si costo < dp[i][j]:
                    dp[i][j] <- costo
		        FinSi
	        FinPara
	    FinPara
    FinPara
    Devolver dp[1][n]
FinProcedimiento
        \end{lstlisting}
    \end{frame}
    \begin{frame}
        \frametitle{Ordenación de multiplicación de matrices: Complejidad}
        \begin{itemize}
            \item Inicializar la matriz nxn tiene como complejidad temporal $\mathcal{O}(n^2)$.
            \item Se ejecuta el bucle externo $n-1$ veces.
            \item El bucle sobre el índice $i$ se ejecuta $n-m+1$ veces. En el peor caso es $\mathcal{O}(n)$.
            \item El bucle sobre el punto de partición $k$ se ejecuta $m-1$ veces. En el peor caso es $\mathcal{O}(n)$.
            \item Cada operación dentro de los bucle tiene complejidad temporal $\mathcal{O}(1)$.
        \end{itemize}
        $\implies$ Los bucles anidados nos dan una complejidad total $\mathcal{O}(n^3)$.
        \begin{equation*}
            T(n) = \mathcal{O}(n^2) + \mathcal{O}(n^3) = \mathcal{O}(n^3)
        \end{equation*}
    \end{frame}

    \section{Optimal Binary Search Tree}
    \begin{frame}
        \frametitle{Optimal Binary Search Tree: Definición}
        Dado un arreglo de elementos \textbf{ordenados} $K[1...n]$ y otro arreglo $V[1...n]$ que indica la frecuencia de cada elemento (también definida como la probabilidad de aparición de cada elemento), 
        Se desea buscar un árbol de búsqueda binaria tal que la métrica:
        \begin{equation*}
            M_K = \sum_{i=1}^{n} (V[i] \times \ell(K[i]))
        \end{equation*}
        Donde $\ell(v)$ representa el nivel de un nodo en el árbol binario, sea mínima
    \end{frame}
    \begin{frame}
        \frametitle{Optimal Binary Search Tree: Solución}
        Se presenta recursión en el problema: Dados $i,j,l : 1\leq i \leq l\leq j\leq n$, podemos construir árboles del subarreglo $K[i...j]$ con raíz $K[l]$
        y subárboles derechos e izquierdos con nodos dentro de los subarreglos $K[i...l-1]$ y $K[l+1...i]$. Se tendría que su métrica puede ser expresada como:
        \begin{equation*}
            M_{i,j} = M_{i,l-1} + M_{l+1,j} + \sum_{k = i}^{j} V[k]
        \end{equation*}
        Dado que se busca el mínimo entre todas estas posibilidades, este sería:
        \begin{equation*}
            M_{i,j}^{\min} = \min_{i\leq l\leq j} \{M_{i,l-1} + M_{l+1,j}\} + \sum_{k = i}^{j} V[k]
        \end{equation*}
        Junto a los casos triviales
        \begin{equation*}
            M_{i,i}^{\min} = V[i] \ \ \ , \ \ \ M_{i,j}^{\min} = 0  \ \ , \ \ \forall j < 1
        \end{equation*}
        Se obtiene la solución para $M^{min} = M_{1,n}$
    \end{frame}

    \begin{frame}[fragile]
        \frametitle{Optimal Binary Search Tree: Pseudocódigo}
        \begin{columns}
            \begin{column}{0.6\textwidth}
                \begin{lstlisting}[basicstyle=\ttfamily\scriptsize]
Procedimiento OptimalBST(K,V)
    dp = construir_dp(V)

    Para d Desde 1 Hasta n-1 Hacer
        Para i Desde 1 Hasta n-d Hacer
            j = d + i
            min <- -inf
            Para l Desde i hasta j Hacer
                m = dp[i-1][l-1] + dp[l][j]
                Si m < min Entonces
                    min = m
                FinSi
            FinPara
            Para k Desde i Hasta j Hacer
                min <- min + V[k-1]
            FinPara
            dp[i-1][j] = min
        FinPara
    FinPara
    Retornar dp[0][n]
FinProcedimiento
                \end{lstlisting}
            \end{column}
            \begin{column}{0.45\textwidth}
                \begin{lstlisting}[basicstyle=\ttfamily\scriptsize]
Procedimiento construir_dp(V)
    n <- V.length
    dp <- Matriz[n+1][n+1]
        todas entradas 0
    Para i Desde 0 Hasta n-1
        dp[i][i+1] = V[i]
    FinPara
    Retornar dp 
FinProcedimiento
                \end{lstlisting}
            \end{column}
        \end{columns}
    \end{frame}
    
    \begin{frame}
        \frametitle{Optimal Binary Search Tree: Complejidad}
        Partimos de la premisa que, dado un arreglo de $n$ entradas, hay $n-m+1$ posibles subarreglos de longitud $m$.
        Por cada subarreglo, se realizan $\mathcal{O}(m)$ operaciones buscando la métrica mínima.
        Entonces:
        \begin{align*}
            T(n) &= \sum_{m = 1}^n (n-m+1) \times cm \\
            &= c(n+1)\sum_{k = 1}^n k - c\sum_{k=1}^n k^2\\
            &= c\frac{n(n+1)(n+2)}{6}\\ 
            =& \mathcal{O}(n^3)
        \end{align*}
        Por lo tanto su complejidad es $\mathcal{O}(n^3)$
    \end{frame}
%    \begin{frame}
%        \frametitle{Optimal Binary Search Tree: Knuth Optimization}
%        El anterior algoritmo se puede optimizar a un problema $\mathcal{O}(n^2)$ si se tiene en cuenta una propiedad muy interesante de 
%        la raíz del árbol que resuelve el problema mínimo:
%        \begin{itemize}
%            \item Si el margen por la derecha se agranda, la raíz no puede moverse a la izquierda
%            \item Si el margen por la izquierda aumenta, la raíz no puede moverse a la
%        \end{itemize}
%%%
%    \end{frame}

\section{Alternating Coin Game}
    \begin{frame}
        \frametitle{Alternating Coin Game: Definición del problema}
        Dado un arreglo $C[1...n]$ que representa los valores de una fila de $n$ monedas, se propone un juego entre dos jugadores con las siguientes reglas:
        \begin{itemize}
            \item Cada jugador toma turnos intercalados
            \item Cada jugador puede sacar una moneda sólo de un extremo de la fila de monedas por turno
            \item Si se saca una moneda, no puede ser regresada a la fila
        \end{itemize}
        El problema consiste en determinar la suma máxima que puede acumular el primer jugador ($S^{\max}$), tomando en cuenta que el adversario \textit{es igual de inteligente que el jugador}
    \end{frame}
    \begin{frame}
        \frametitle{Alternating Coin Game: Solución}
        Como sabemos que el oponente es igual de inteligente que nosotros, asumimos que el oponente \textbf{siempre tratará de tomar la moneda que nos conduza a tener la menor suma posible}.\\
        Supongamos que han transcurrido algunos turnos y tenemos el subarreglo $C[i,...,j]$ del arreglo original de monedas. Tenemos dos casos:
        \begin{itemize}
            \item Tomamos $C[i]$: Entonces el oponente tiene dos opciones, tomar C[i+1] o C[j]. Dado que el oponente buscará tomar la alternativa que nos conduzca a tomar el mínimo valor posible,
            entonces el máximo valor que podemos recolectar es:
            \begin{equation*}
                S_{i,j} = C[i] + \min\{S_{i+2,j},S_{i+1,j-1}\}
            \end{equation*}
            \item Tomamos $C[j]$: Análogamente al caso anterior, el oponente hará su jugada y tendremos que el valor máximo recolectable es
            \begin{equation*}
                S_{i,j} = C[j] + \min\{S_{i,j-2},S_{i+1,j-1}\}
            \end{equation*}
        \end{itemize}
    \end{frame}
    \begin{frame}
        \frametitle{Alternating Coin Game: Solución}
        El resultado que maximize nuestra ganancia es el máximo entre estos dos casos. Así:
        \begin{equation*}
            S_{i,j}^{\max} = \max\{ C[i] + \min\{S_{i+2,j}^{\max},S_{i+1,j-1}^{\max}\}, C[j] + \min\{S_{i,j-2}^{\max},S_{i+1,j-1}^{\max}\} \}
        \end{equation*}
        Junto a los casos triviales
        \begin{align*}
            &S_{i,i}^{\max} = C[i] \\
            &S_{i,j}^{\max} = 0 \ \ , \ \  \forall j < i
        \end{align*}
        Tendríamos que la respuesta al problema se obtendría computando:
        \begin{equation*}
            S^{\max} = S_{1,n}^{\max}
        \end{equation*}
    \end{frame}

    \begin{frame}[fragile]
        \frametitle{Alternating Coin Game: Pseudocódigo}
        \begin{columns}
            \begin{column}{0.6\textwidth}
                \begin{lstlisting}[basicstyle=\ttfamily\scriptsize]
Procedimiento maxGameResult(C)
    n <- C.length
    dp <- Matriz[n+1][n+1]
    construir(dp)
    Para d Desde 2 Hasta n Hacer
        j = d+i
        Para i Desde 0 Hasta n-d Hacer
            s_1 = C[i] + 
            min(dp[i+2][j],dp[i+1][j-1])
            s_2 = C[j-1] + 
            min(dp[i][j-2],dp[i+1][j-1])
            dp[i][j] = max(s_1,s_2)
        FinPara
        Retornar dp[0][n]
    FinPara
FinProcedimiento
                \end{lstlisting}
            \end{column}
            \begin{column}{0.5\textwidth}
                \begin{lstlisting}[basicstyle=\ttfamily\scriptsize]
Procedimiento construir_dp(dp)
    n <- K.length
    Para i Desde 0 Hasta n
        Para j Desde 0 Hasta i
            dp[i][j] = 0
        FinPara
        Si i < n Entonces
            dp[i][i+1] = V[i]
        FinSi
    FinPara
    Retornar dp 
FinProcedimiento    
                \end{lstlisting}
            \end{column}
        \end{columns}
    \end{frame}

    \begin{frame}
        \frametitle{Alternating Coin Game: Complejidad}
        La cantidad de subarreglos obtenibles de $C[1...n]$ de longitud $m$ es $n-m+1$
        Por cada subarreglo, se ejecutan $\mathcal{O}(1)$ operaciones para obtener la solución.
        Así:
        \begin{align*}
            T(n)&=\sum_{m = 1}^{n} (n-m+1) \times c \\
            &= c \left( n(n+1) - \frac{n(n+1)}{2} \right) \\
            &= c\frac{n(n+1)}{2}\\
            &= \mathcal{O}(n^2)
        \end{align*}
        Finalmente, el algoritmo tiene complejidad $\mathcal{O}(n^2)$
    \end{frame}
\end{document}